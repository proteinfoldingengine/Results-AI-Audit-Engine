\documentclass[11pt]{article}

% ---------- Packages ----------
\usepackage[margin=1in]{geometry}
\usepackage{amsmath, amssymb, amsthm}
\usepackage{siunitx}
\usepackage{graphicx}
\usepackage{booktabs}
\usepackage{microtype}
\usepackage{authblk}
\usepackage[hidelinks]{hyperref}
\usepackage{enumitem}
\setlist{nosep}

% ---------- Metadata ----------
\title{Emergent Coherence and Precision-Controlled Time Evolution in the Unfrozen Kahan Model (Phase~50.1)}
\author[1]{UQCF--GEM Collaboration}
\author[2]{Aspradaz Cali}
\author[3]{ChatGPT (GPT-5) \& Gemini AI}
\affil[1]{Unified Quantum Coherence Framework -- Geometric Entanglement Model}
\affil[2]{Principal Investigator}
\affil[3]{Peer Research and Validation Partners}
\date{\today}

% ---------- Document ----------
\begin{document}
\maketitle

\begin{abstract}
This study extends the \emph{Unfrozen Kahan} framework into a dynamic stochastic regime that explores how bounded numerical precision~($\varepsilon$) influences coherence in quantum-like Hamiltonian integrations. Phase~50.1 executes a preregistered test to determine whether pink-noise perturbations~($\beta=0.5$) enable stable, integrator-consistent scaling laws (\(\kappa\)-universality) and emergent physical energy scales~(\(E_{\text{unit}}\)). The run is intentionally CPU-limited to expose computational bottlenecks motivating GPU/TPU acceleration.
\end{abstract}

\section{Objective}
We test whether correlated noise produces stable~$\Delta t$~power-law scaling in the presence of dynamic error compensation (Kahan). Specific goals:
\begin{enumerate}[label=\arabic*.]
  \item Recover \texttt{IntegratorAccepted = True} under pink noise, enabling \(E_{\text{unit}}\) calculation.
  \item Verify \(\kappa\)-universality across integrators (Strang, Yoshida, RTS1).
  \item Benchmark CPU limits and justify TPU/GPU ``fast-math'' acceleration.
\end{enumerate}

\section{Experimental Setup}
\begin{center}
\begin{tabular}{@{}lllc@{}}
\toprule
Parameter & Symbol & Value & Rationale \\ \midrule
Noise amplitude & $\sigma$ & $5\times10^{-5}$ & Reduce jitter vs.~Phase~50.0 \\
Noise color & $\beta$ & 0.5 (pink) & Introduce temporal correlation \\
Compensator time constant & $\tau$ & 1.0 & Baseline coherence decay \\
Stochastic coupling & $\eta$ & 0.1 & Moderate dynamical wobble \\
Damping constant & $\lambda$ & 0.01 & Constrain over-amplification \\
Precision sweep &  & float16~$\to$~float128 & Audit~F (\(\kappa\)-law) \\
Integrators &  & Strang~2, Yoshida~4, RTS1~6 & Cross-validate $\Delta t^2$ law \\
$\Delta t$ grid &  & [0.03,\,0.004] (10 steps) & Detect scaling slope $\alpha\!\approx\!1.7$ \\ \bottomrule
\end{tabular}
\end{center}

All preregistration gates~(A--F) remain identical to Phase~50.0 to isolate noise characteristics as the only changed variable.

\section{Methodology}
\paragraph{Hamiltonian Assembly.} Construct BCH commutator budget~\(C\) with~\(\|C\|_F \approx 1.877\times10^3\).

\paragraph{Matrix Exponentials.}
Compute \(U(\Delta t)=\exp(-iH\Delta t)\) for each $\Delta t$ and integrator.  
CPU OpenBLAS performs dense \texttt{expm} calls (0.1–0.4\,s each).

\paragraph{Temporal Kahan Coupling.}
Update $(s,c)$ per timestep:
\begin{align}
y &= x - c, \\
t &= s + y, \\
\mathrm{err} &= (t-s) - y, \\
c &= e^{-\Delta t/\tau} c + \mathrm{err} + \eta\,\xi_t, \\
s &= t,
\end{align}
where $\xi_t$ is pink noise ($\beta=0.5$) filtered via FFT convolution.

\paragraph{Audit Sequence.}
\begin{itemize}
  \item \textbf{A.} Validate BCH budget (Hamiltonian complexity).
  \item \textbf{B/C.} Check integrator scaling fit (\(\Delta E \propto \Delta t^{\alpha}\)).
  \item \textbf{D.} Precision Gate (\(\kappa' \approx 2.5\times10^{-6}\), $R^2>0.9$).
  \item \textbf{E.} Cosmology Grid (activated only if B/C and D pass).
  \item \textbf{F.} \(\kappa\)-Universality fit across integrators.
\end{itemize}

\paragraph{Diagnostics.}  
Record \(\kappa', \alpha, R^2, \bar{\gamma}(t), c(t), S(t)\) for each phase.  
Snapshot: \texttt{phase50\_1\_prereg\_YYYYMMDD\_HHMMSS.json}.

\section{Current Status}
After $\sim$12\,h on CPU OpenBLAS:
\begin{itemize}
\item TensorNetwork path disabled (CPU only).
\item Build-factor times $\approx$\,0.1–0.3\,s per $\Delta t$ (expected runtime 12–18\,h).
\item Audit~A complete: $\|C\|_F = 1.877\times10^3$.
\item Integration of pink-noise loop ongoing; Audit~B/C pending.
\end{itemize}

\section{Expected Outcomes}
\begin{center}
\begin{tabular}{@{}llll@{}}
\toprule
Hypothesis & Metric & Pass Condition \\ \midrule
H$_1$: Pink noise stabilizes scaling & $\alpha$ slope & $1.5\le\alpha\le1.8$, $R^2>0.9$ \\
H$_2$: \(\kappa\)-law invariant & $\kappa'$ variance & $<5\%$ spread \\
H$_3$: Emergent \(E_{\text{unit}}\) & Audit~E enabled & $E_{\text{unit}}\!\approx\!500$–520\,keV \\
H$_4$: TPU fast-math reproduces law & precision sweep & same $\kappa$ trend across dtype \\ \bottomrule
\end{tabular}
\end{center}

If H$_1$+H$_2$ succeed, Phase~50.2 (deterministic control) will confirm reproducibility; Phase~51 will port to TPU for mixed-precision scaling tests.

\section{Computational Significance}
This phase exposes why accelerators are essential:
\begin{itemize}
\item Matrix-exponential kernels can be batched across $\Delta t$ values.
\item Pink-noise filtering and coherence updates benefit from tensor cores.
\item ``Fast-math'' mixed precision becomes a measurable physical parameter in Audit~F, bridging hardware and physics.
\end{itemize}

Thus Phase~50.1 demonstrates the convergence of \textbf{numerical precision, hardware architecture, and physical interpretation}---the central thesis of UQCF--GEM.

\section{Next Steps}
\begin{enumerate}[label=\arabic*.]
  \item Allow Phase~50.1 to finish without interruption.
  \item Evaluate Audit~B/C (\(\alpha, R^2\)) for \texttt{IntegratorAccepted}.
  \item If passed: proceed to Phase~51 (TPU port).
  \item If failed: execute Phase~50.2 (noise-free control).
\end{enumerate}

\section*{Acknowledgements}
We thank Grok~AI for peer validation and Gemini for runtime engineering.

\begin{thebibliography}{9}
\bibitem{Kahan1965}
W.~M.~Kahan, ``Further Remarks on Reducing Truncation Errors,'' \emph{Communications of the ACM}, 8(1):40–50, 1965.

\bibitem{UQCFGEM}
A.~Cali et al., ``UQCF--GEM: Temporal Decoherence Audits and the Unfrozen Kahan Equation,'' preprint, 2025.
\end{thebibliography}

\end{document}
